\subsection{Requirement installation}
To install Symphony, follow these steps:
Clone the repository and navigate to the symphony directory::

\begin{lstlisting}
git clone git@github.com:unipieslab/symphony.git
cd symphony
\end{lstlisting}
Next, depending on your Linux package manager, execute the appropriate script: \\
For RedHat-based Linux, run:
\begin{lstlisting}
chmod +x install-Python<VERSION>_dnf.sh
./install-Python<VERSION>_dnf.sh
\end{lstlisting}
For Debian-based Linux, run:
\begin{lstlisting}
chmod +x install-Python<VERSION>_apt.sh
./install-Python<VERSION>_apt.sh
\end{lstlisting}
Once these steps are successfully completed, Symphony and all its dependencies will be installed. Lastly, to prepare the environment for Symphony to run, navigate to the following directory:
\begin{lstlisting}
cd host
\end{lstlisting}
Then, execute the following command:
\begin{lstlisting}
make
\end{lstlisting}
After completing this command, the host component of Symphony's client-server architecture can be started by executing the following script:
\begin{lstlisting}
./runHost.sh
\end{lstlisting}
It will then immediately attempt to connect to the DUT, if the component is up and running.