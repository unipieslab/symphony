\subsection{Experiment start}
A monitoring routine that examines the behavior of a device is a complex and time-consuming procedure to develop. Such a routine must be able to retrieve integral information such as the following:
\begin{itemize}
  \item The temperature of the system.
  \item The current-voltage value (on undervolt experiments).
  \item Determine the overall health of the system
  \item Possible SDCs that occurred
\end{itemize}
Moreover, such a routine must take action in scenarios where human interference is otherwise required. One such scenario is if the device had a kernel panic and must reset to continue the experiment. 

To relieve the user from developing a routine like this, Symphony provides a built-in function that considers all the mentioned scenarios and more.

\subsubsection{Function signature}

\begin{lstlisting}
def experiment_start(self)
\end{lstlisting}

\subsubsection{Description}
\begin{lstlisting}[mathescape=true, keywordstyle=\color{black}]
Monitors the DUT system for any possible 
issues, take action when necessary, and save 
JSON formatted files that contain data from DUT (for post-
processing and analysis).

This routine, however, requires several user-implemented 
functions to perform with the expected behavior 
(see section with callbacks). 
\end{lstlisting}

\subsubsection{Parameters}
\begin{lstlisting}[mathescape=true, keywordstyle=\color{black}]
No parameters are required.
\end{lstlisting}

\subsubsection{Returns}
\begin{lstlisting}[mathescape=true, keywordstyle=\color{black}]
Nothing is returned
\end{lstlisting}


\subsubsection{Usage Example}
\begin{lstlisting}
tester = Tester_Shell() # An instance of Symphony (Host)

# User-defined functions...
# Code to make adjustments to the DUT system ... 

tester.experiment_start()

\end{lstlisting}

