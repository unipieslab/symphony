\subsection{simple\_remote\_execute}

\subsubsection{Function signature}
\begin{lstlisting}
def simple_remote_execute(self, cmd: str, times: int, ret_imediate: bool) -> list
\end{lstlisting}

\subsubsection{Description}
\begin{lstlisting}[mathescape=true, keywordstyle=\color{black}, showstringspaces=false]
Is a simpler version of the corresponding remote_execute 
routine. The only difference between these two routines is 
that the simple_remote_execute routine requires fewer 
parameters than remote_execute does, making it easier to 
use in some cases where the extra parameters are 
redundant, like executing a command whose execution time 
is only a few seconds. If three attempts of execution have 
been attempted, then, the DUT is ordered to do a hard 
reset
\end{lstlisting}

\subsubsection{Parameters}
\begin{lstlisting}[mathescape=true, keywordstyle=\color{black}]
$\textbf{cmd}$: Represents a string that is the command to execute

$\textbf{times}$: An integer representing the number of times the DUT 
system must execute the requested command. 

$\textbf{ret\_imediate}$: Represents a logical value that allows the
users to force the routine to return immediately after
encountering any (communication-related) error.

\end{lstlisting}

\subsubsection{Returns}
\begin{lstlisting}[mathescape=true, keywordstyle=\color{black}]
Returns a list holding the results of the requested
command. The list contains an element number equal to the
number of times the requested command was executed
\end{lstlisting}


\subsubsection{Usage Example}
\begin{lstlisting}

tester = Tester_Shell() # An instance of Symphony (Host)
ls_cmd_timetout = 2 # seconds

tester.simple_remote_execute("ls", 2, False)

\end{lstlisting}
