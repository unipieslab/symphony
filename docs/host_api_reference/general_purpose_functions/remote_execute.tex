\subsection{remote\_execute}

\subsubsection{Function signature}
\begin{lstlisting}
def remote_execute(self, cmd: str, cmd_timeout_s: int, net_timeout_s: int, dmesg_index: int, times: int, ret_imediate: bool) -> list:
\end{lstlisting}

\subsubsection{Description}
\begin{lstlisting}[mathescape=true, keywordstyle=\color{black}, showstringspaces=false]
Executes user-requested bash commands to the DUT system. 
In case of any (communication-related) error that may 
encountered during the execution of the requested command, 
there will be attempts to finish the execution. If three 
attempts of execution have been attempted, then, the DUT 
is ordered to do a hard reset. 
\end{lstlisting}

\subsubsection{Parameters}
\begin{lstlisting}[mathescape=true, keywordstyle=\color{black}]
$\textbf{cmd}$: Represents a string that is the command to execute
$\textbf{cmd\_timeout\_s}$: An integer representing the expected number 
of seconds it takes from DUT to execute the requested 
command. Timeouts like this can be estimated using the 
routine "estimate_timeouts" (see section)
$\textbf{net\_timeout\_s}$: Represents the expected delay due to the
network infrastructure in seconds. Users may choose their
value of preference, but Symphony offers the built-in
constant value $\textbf{NETWORK\_TIMEOUT\_SEC}$, which can be used
instead.

$\textbf{dmesg\_index}$: This parameter represents an integer that 
specifies the position where the last command left the 
dmesg file. The user may use this function with$\textbf { dmesg = 0}$, 
as the genuine usage of this parameter matters only in the 
internal implementation of the symphony.

$\textbf{times}$: An integer representing the number of times the DUT 
system must execute the requested command. 

$\textbf{ret\_imediate}$: Represents a logical value that allows the
users to force the routine to return immediately after
encountering any (communication-related) error.
\end{lstlisting}


\subsubsection{Returns}
\begin{lstlisting}[mathescape=true, keywordstyle=\color{black}]
Returns a list holding the results of the requested 
command. The list contains an element number equal to the 
number of times the requested command was executed. 
\end{lstlisting}


\subsubsection{Usage Example}
\begin{lstlisting}
tester = Tester_Shell() # An instance of Symphony (Host)
ls_cmd_timetout = 2 # seconds

tester.remote_execute("ls", ls_cmd_timeout, Tester_Shell_Constants. NETWORK_TIMEOUT_SEC.value, 0, 2, False)

\end{lstlisting}
