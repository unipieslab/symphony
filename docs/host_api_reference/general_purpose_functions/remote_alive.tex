\subsection{remote\_alive}

\subsubsection{Function signature}
\begin{lstlisting}
def remote_alive(self, net_timeout_s: int, ret_imediate: bool) -> bool
\end{lstlisting}

\subsubsection{Description}
\begin{lstlisting}[mathescape=true, keywordstyle=\color{black}, showstringspaces=false]
Check whether the DUT system is down. In case of any 
(communication-related) error that may encountered during 
the execution of the requested command, there will be 
attempts to finish the execution. If three attempts of 
execution have been attempted, then, the DUT
is ordered to do a hard reset.
\end{lstlisting}
\subsubsection{Parameters}
\begin{lstlisting}[mathescape=true, keywordstyle=\color{black}]

$\textbf{net\_timeout\_s}$: Represents the expected delay due to the 
network infrastructure in seconds. Users may choose their 
value of preference, but Symphony offers the built-in 
constant value $\textbf{NETWORK\_TIMEOUT\_SEC}$, which can be used 
instead.  

$\textbf{ret\_imidiate}$: Represents a logical value that allows the 
users to force the routine to return immediately after 
encountering any (communication-related) error.

\end{lstlisting}

\subsubsection{Returns}
\begin{lstlisting}[mathescape=true, keywordstyle=\color{black}]
Either $\textbf{True}$, in case the DUT is up, or 
$\textbf{False}$ in case it is down.
\end{lstlisting}


\subsubsection{Usage Example}
\begin{lstlisting}
tester = Tester_Shell() # An instance of Symphony (Host)

tester.remote_alive(Tester_Shell_Constants.NETWORK_TIMEOUT_SEC.value, False) # Check whether the DUT system is visible on the network.
\end{lstlisting}
