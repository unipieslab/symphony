\subsection{Undervolt experiment}
Another routine that is often vital is a routine that performs an undervolting experiment. Such a routine must take care of a few essential actions listed below:
\begin{itemize}
  \item To reduce the voltage of the system.
  \item Evaluate the device's overall health in an (undervolted) voltage value.
  \item Execute each benchmark for the instructed total time.
\end{itemize}
Symphony has an existing implementation of such a routine, explained in the following definition.

\subsubsection{Function signature}
\begin{lstlisting}
def target_perform_undervolt_test(self)
\end{lstlisting}

\subsubsection{Description}
\begin{lstlisting}[mathescape=true, keywordstyle=\color{black}]
Performs an undervolted experiment for some
minutes, specified in the JSON. This routine makes the 
undervolting process and benchmark execution easy and 
automatic. 

This routine, however, requires several user-implemented 
functions to perform with the expected behavior 
(see section with callbacks). 
\end{lstlisting}

\subsubsection{Parameters}
\begin{lstlisting}[mathescape=true, keywordstyle=\color{black}]
No parameters are required.
\end{lstlisting}

\subsubsection{Returns}
\begin{lstlisting}[mathescape=true, keywordstyle=\color{black}]
Nothing is returned
\end{lstlisting}


\subsubsection{Usage Example}
\begin{lstlisting}
tester = Tester_Shell() # An instance of Symphony (Host)
# User-defined functions...
# Code to make adjustments to the DUT system ... 

tester.target_perform_undervolt_test()
\end{lstlisting}
