\subsection{Find minimum operating voltage}
An essential part of an undervolting experiment is to find, through experimenting, a voltage value that is the minimum voltage in which the device being undervolted is operating correctly. To save time, Symphony has a built-in function that does exactly that.

\subsubsection{Function signature}

\begin{lstlisting}
def auto_undervolt_characterization(self, duration_per_bench_min: int, characterization_id: str) -> int
\end{lstlisting}

\subsubsection{Parameters}
\begin{lstlisting}[mathescape=true]
$\textbf{duration\_per\_bench\_min}$: An integer representing the number 
of minutes each benchmark should be on a specific 
(undervolted) voltage value.

$\textbf{characterization\_id}$: This field represents a string that 
corresponds to an identification for the test.

\end{lstlisting}

\subsubsection{Returns}
\begin{lstlisting}[mathescape=true]
Typically returns an integer representing the value of the 
requested voltage (aka Vmin). Otherwise, nothing is 
returned, and the user must examine the logs to determine 
Vmin.
\end{lstlisting}

\subsubsection{Usage Example}
\begin{lstlisting}
tester = Tester_Shell() # An instance of Symphony (Host)
undervolt_cmd= "Bash command for undervolting"
v_nominal = N # Where N is the initial voltage upon boot (in hexadecimal)

res = tester.auto_undervolt_characterization(n_nominal,undervolt_cmd)  # Find the Minimum operating voltage
\end{lstlisting}
(\textbf{To run the above example, there are a few preliminary steps to follow. Refer to \autoref{sec:getting_started}})